\documentclass[journal,12pt,twocolumn]{IEEEtran}
\usepackage{amsthm}
\usepackage{gensymb}
\usepackage{setspace}
\singlespacing
\usepackage[cmex10]{amsmath}
\usepackage{bm}

\usepackage{cases}
\usepackage{mathrsfs}
\usepackage{cite}
\usepackage{stfloats}
\usepackage{mathtools}
\usepackage[breaklinks=true]{hyperref}
\usepackage{graphicx}
\usepackage{subfig}
\usepackage{txfonts}
\usepackage{longtable}
\usepackage{multirow}
\usepackage{tfrupee}
\usepackage{enumitem}
\usepackage{tikz}
\usepackage{steinmetz}
\usepackage{verbatim}
\usepackage{circuitikz}
\usepackage{tkz-euclide}





\usetikzlibrary{calc,math}
\usepackage{listings}
    \usepackage{color}                                            %%
    \usepackage{array}                                            %%
    \usepackage{longtable}                                        %%
    \usepackage{calc}                                             %%
    \usepackage{multirow}                                         %%
    \usepackage{hhline}                                           %%
    \usepackage{ifthen}                                           %%
    \usepackage{lscape}     
\usepackage{multicol}
\usepackage{chngcntr}

\DeclareMathOperator*{\Res}{Res}

\renewcommand\thesection{\arabic{section}}
\renewcommand\thesubsection{\thesection.\arabic{subsection}}
\renewcommand\thesubsubsection{\thesubsection.\arabic{subsubsection}}

\renewcommand \thesectiondis{\arabic{section}}
\renewcommand\thesubsectiondis{\thesectiondis.\arabic{subsection}}
\renewcommand\thesubsubsectiondis{\thesubsectiondis.\arabic{subsubsection}}


\hyphenation{op-tical net-works semi-conduc-tor}
\def\inputGnumericTable{}                                 %%

\lstset{
%language=C,
frame=single, 
breaklines=true,
columns=fullflexible
}
\begin{document}

\newcommand{\BEQA}{\begin{eqnarray}}
\newcommand{\EEQA}{\end{eqnarray}}
\newcommand{\define}{\stackrel{\triangle}{=}}
\bibliographystyle{IEEEtran}
\raggedbottom
\setlength{\parindent}{0pt}
\providecommand{\mbf}{\mathbf}
\providecommand{\pr}[1]{\ensuremath{\Pr\left(#1\right)}}
\providecommand{\qfunc}[1]{\ensuremath{Q\left(#1\right)}}
\providecommand{\sbrak}[1]{\ensuremath{{}\left[#1\right]}}
\providecommand{\lsbrak}[1]{\ensuremath{{}\left[#1\right.}}
\providecommand{\rsbrak}[1]{\ensuremath{{}\left.#1\right]}}
\providecommand{\brak}[1]{\ensuremath{\left(#1\right)}}
\providecommand{\lbrak}[1]{\ensuremath{\left(#1\right.}}
\providecommand{\rbrak}[1]{\ensuremath{\left.#1\right)}}
\providecommand{\cbrak}[1]{\ensuremath{\left\{#1\right\}}}
\providecommand{\lcbrak}[1]{\ensuremath{\left\{#1\right.}}
\providecommand{\rcbrak}[1]{\ensuremath{\left.#1\right\}}}
\theoremstyle{remark}
\newtheorem{rem}{Remark}
\newcommand{\sgn}{\mathop{\mathrm{sgn}}}
\providecommand{\abs}[1]{\vert#1\vert}
\providecommand{\res}[1]{\Res\displaylimits_{#1}} 
\providecommand{\norm}[1]{\lVert#1\rVert}
%\providecommand{\norm}[1]{\lVert#1\rVert}
\providecommand{\mtx}[1]{\mathbf{#1}}
\providecommand{\mean}[1]{E[ #1 ]}
\providecommand{\fourier}{\overset{\mathcal{F}}{ \rightleftharpoons}}
%\providecommand{\hilbert}{\overset{\mathcal{H}}{ \rightleftharpoons}}
\providecommand{\system}{\overset{\mathcal{H}}{ \longleftrightarrow}}
	%\newcommand{\solution}[2]{\textbf{Solution:}{#1}}
\newcommand{\solution}{\noindent \textbf{Solution: }}
\newcommand{\cosec}{\,\text{cosec}\,}
\providecommand{\dec}[2]{\ensuremath{\overset{#1}{\underset{#2}{\gtrless}}}}
\newcommand{\myvec}[1]{\ensuremath{\begin{pmatrix}#1\end{pmatrix}}}
\newcommand{\mydet}[1]{\ensuremath{\begin{vmatrix}#1\end{vmatrix}}}
\numberwithin{equation}{subsection}
\makeatletter
\@addtoreset{figure}{problem}
\makeatother
\let\StandardTheFigure\thefigure
\let\vec\mathbf
\renewcommand{\thefigure}{\theproblem}
\def\putbox#1#2#3{\makebox[0in][l]{\makebox[#1][l]{}\raisebox{\baselineskip}[0in][0in]{\raisebox{#2}[0in][0in]{#3}}}}
     \def\rightbox#1{\makebox[0in][r]{#1}}
     \def\centbox#1{\makebox[0in]{#1}}
     \def\topbox#1{\raisebox{-\baselineskip}[0in][0in]{#1}}
     \def\midbox#1{\raisebox{-0.5\baselineskip}[0in][0in]{#1}}
\vspace{3cm}
\title{Assignment6}%number
\author{CS20Btech11035 -NYALAPOGULA MANASWINI}
\maketitle
\newpage
\bigskip

\renewcommand{\thefigure}{\theenumi}
\renewcommand{\thetable}{\theenumi}
Download python code from 
\begin{lstlisting}
https://github.com/N-Manaswini23/assignment6/blob/main/python%20codes/assignment6.py
\end{lstlisting}
%
Download latex code from 
\begin{lstlisting}
https://github.com/N-Manaswini23/assignment6/blob/main/assignment6.tex
\end{lstlisting}
%

\section*{GATE 2019 ME set-2 QUESTION 28}
The variable $x$ takes a value between $0$ and $10$ with uniform probability distribution. The variable $y$ takes a value between $0$ and $20$ with uniform probability distribution.The probability that sum of variables ($x+y$) being greater than $20$ is

\section*{SOLUTION}
Given variable $x$ takes a value between $0$ and $10$\\
variable $y$ takes a value between $0$ and $20$


\begin{figure}[htb!]
\begin{center}
\includegraphics[width=1\columnwidth]{assignment6.png}
\end{center}
\caption{Graph}
\label{graph:1}
\end{figure}

From graph \eqref{graph:1}
\begin{align}
\Pr(x+y>20)&=\frac{\mbox{Area of shaded region}}{\mbox{Area of rectangle}} \label{1} \\
&=\frac{\frac{1}{2}\times 10 \times 10}{10 \times 20} \\
&=\frac{1}{4}\\
\therefore \Pr(x+y>20)&=\frac{1}{4}=0.25
\end{align}



\end{document}